%==============================================================================
% Sjabloon onderzoeksvoorstel bachelorproef
%==============================================================================
% Gebaseerd op LaTeX-sjabloon ‘Stylish Article’ (zie voorstel.cls)
% Auteur: Jens Buysse, Bert Van Vreckem
%
% Compileren in TeXstudio:
%
% - Zorg dat Biber de bibliografie compileert (en niet Biblatex)
%   Options > Configure > Build > Default Bibliography Tool: "txs:///biber"
% - F5 om te compileren en het resultaat te bekijken.
% - Als de bibliografie niet zichtbaar is, probeer dan F5 - F8 - F5
%   Met F8 compileer je de bibliografie apart.
%
% Als je JabRef gebruikt voor het bijhouden van de bibliografie, zorg dan
% dat je in ``biblatex''-modus opslaat: File > Switch to BibLaTeX mode.

\documentclass{voorstel}

\usepackage{lipsum}

%------------------------------------------------------------------------------
% Metadata over het voorstel
%------------------------------------------------------------------------------

%---------- Titel & auteur ----------------------------------------------------

% TODO: geef werktitel van je eigen voorstel op
\PaperTitle{Titel voorstel}
\PaperType{Onderzoeksvoorstel Bachelorproef 2020-2021}

% TODO: vul je eigen naam in als auteur, geef ook je emailadres mee!
\Authors{Matthias Tison\textsuperscript{1}} % Authors
\CoPromotor{Arvid De Meyer\textsuperscript{2} (Codifly)}
\affiliation{\textbf{Contact:}
  \textsuperscript{1} \href{mailto:matthias.tison.w0715@student.hogent.be}{matthias.tison.w0715@student.hogent.be};
  \textsuperscript{2} \href{mailto:arvid.de.meyer@codifly.be}{arvid.de.meyer@codifly.be};
}

%---------- Abstract ----------------------------------------------------------

\Abstract{\\
    Dit werk draagt bij tot het opzetten van een basis voor stabiele JavaScript backend projecten in Node.js met TypeScript integratie. De opkomst van microservice architectuur in vele bedrijven reikt de hand naar het alom populaire Node.js als programmeer taal. Microservice principe steunt op het feit van meerdere onafhankelijke services in een cluster te hebben die samen een de gehele backend vormen. Dit principe roept om een consistente project structuur onder de services, zeker wanneer een microservice architectuur al gouw tien tot vijftien services kan omvatten. \\
    Het onderzoek bestaat eruit een stabiele en ontwikkelaars vriendelijke basis te leggen voor backend services in Node.js. Dit onderzoek focust zich op REST API's als backend service. Hierbij wordt de volledige `tech stack` bepaald en uitgelegd, met daarbij de nodige integratie in de project structuur. Verscheidene onderdelen van de `tech stack` worden bepaald op basis van vergelijking tussen verschillende technologieën. Volgend onderzoeksvoorstel stelt de voorstudie op van een backend oplossing in Node.js met Typescript integratie en wat verwacht wordt van de finale uitwerking. \\
    Uit dit onderzoek wordt getracht een robuuste, stabiele en ontwikkelaars vriendelijke omgeving op te zetten voor backend oplossingen in Node.js met Typescript integratie. Dit volgens de juiste principes voor het opzetten van dergelijke API, met daarbij de meest toepasselijke technologieën op basis van gegrond onderzoek. Aansluitend zou er CI  (continuous integration) en CD (continuous development) voorzien kunnen worden, zodat het release proces voor dergelijke projecten ook toevoeging heeft tot stabiliteit.
\\}

%---------- Onderzoeksdomein en sleutelwoorden --------------------------------
% TODO: Sleutelwoorden:
%
% Het eerste sleutelwoord beschrijft het onderzoeksdomein. Je kan kiezen uit
% deze lijst:
%
% - Mobiele applicatieontwikkeling
% - Webapplicatieontwikkeling
% - Applicatieontwikkeling (andere)
% - Systeembeheer
% - Netwerkbeheer
% - Mainframe
% - E-business
% - Databanken en big data
% - Machineleertechnieken en kunstmatige intelligentie
% - Andere (specifieer)
%
% De andere sleutelwoorden zijn vrij te kiezen

\Keywords{Applicatieontwikkeling -- Backend -- REST API -- Node.js -- TypeScript} % Keywords
\newcommand{\keywordname}{Sleutelwoorden} % Defines the keywords heading name

%---------- Titel, inhoud -----------------------------------------------------

\begin{document}

\flushbottom % Makes all text pages the same height
\maketitle % Print the title and abstract box
\tableofcontents % Print the contents section
\thispagestyle{empty} % Removes page numbering from the first page

%------------------------------------------------------------------------------
% Hoofdtekst
%------------------------------------------------------------------------------

% De hoofdtekst van het voorstel zit in een apart bestand, zodat het makkelijk
% kan opgenomen worden in de bijlagen van de bachelorproef zelf.
%---------- Inleiding ---------------------------------------------------------

\section{Introductie} % The \section*{} command stops section numbering
\label{sec:introductie}

%\begin{itemize}
%  \item de probleemstelling en context
%  \item de motivatie en relevantie voor het onderzoek
%  \item de doelstelling en onderzoeksvraag/-vragen
%\end{itemize}

Het opkomen van de microservice architectuur heeft dit ook doen binnen dringen in talrijke bedrijven. Vele slagen erin om deze strategie correct te implementeren, maar vele lijden ook onder de nieuwe manier van werken met verschillende onafhankelijke services.
Het gebruik van dit soort architectuur leunt er zich ook toe om het werk en `ownership` van de verschillende services te verdelen onder teams. Elk team werkt op zijn eigen manier onder, idealiter, een `agile` ontwikkelingsproces aan de toegekende service. Een microservice architectuur omvat ook snel 10 tot 15 services, waardoor teams ook sneller een bepaald aantal van de services beheren. \\
Hieruit kan geconcludeerd worden dat er nood is aan het opbouwen van standaarden over heen teams. Het opleggen van een bepaalde standaard brengt een vorm van consistentie en meer consistentie lijdt uiteindelijk tot stabiliteit. Dit onderzoek zal bijdragen tot het creëren van een doorgronde opzet als basis tot stabiele backend services in Node.js, met daarbij specifiek de focus op REST API als microservice. \\
Voor het doorvoeren van bepaalde standaarden moet er gegrond onderzoek gevoerd worden die kan onderbouwen waarom de standaard in kwestie gehanteerd wordt. Hierbij kan afgevraagd worden waarom een Typescript integratie benefiet zou hebben in Node.js en hoe zo een integratie gebeurd? Welke stappen worden uitgevoerd om een bepaald niveau van `code quality` te bekomen? Wat zijn de meest prominente technologieën als `tech stack` voor REST API in Node.js? Hoe kunnen REST principes correct toegepast worden in correlatie met de `tech stack` en projectstructuur?

%---------- Stand van zaken ---------------------------------------------------

\section{State-of-the-art}
\label{sec:state-of-the-art}

% Voor literatuurverwijzingen zijn er twee belangrijke commando's:
% \autocite{KEY} => (Auteur, jaartal) Gebruik dit als de naam van de auteur
%   geen onderdeel is van de zin.
% \textcite{KEY} => Auteur (jaartal)  Gebruik dit als de auteursnaam wel een
%   functie heeft in de zin (bv. ``Uit onderzoek door Doll & Hill (1954) bleek
%   ...'')

\subsection{Waarom Node.js?}

Node.js is sinds enkele jaren één van de standaarden voor backend services, samen met grootheden zoals Java en Python. Sinds de eerst release in 2009 is het steeds volwassener geworden en heeft daarbij een stand vaste community opgebouwd. De JavaScript runtime environment maakt het mogelijk om de geliefde frontend web taal ook naar de backend te brengen. \\
Met de opkomst van microservices biedt Node.js een geschikte omgeving. Door de makkelijke schaalbaarheid en native ondersteuning voor HTTP aanvragen leunt het zich perfect voor gebruik als REST API. Met native JSON serialisatie en de-serialisatie biedt het een veel gebruikte standaard voor client/server-architectuur.

\subsection{TypeScript integratie}

JavaScript geeft de mogelijkheid om snel en vrij te ontwikkelen door zijn dynamisch typering, wat betekend dat de types van gedefinieerde variabelen niet gekend zijn tot in runtime. Naast dynamisch is JavaScript ook zwak getypeerd en betekend dat variabelen niet gebonden zijn tot één data type. Dit staat toe om variabelen verschillende data types, zoals string of number, te geven doorheen de code. \\
Voor veel ontwikkelaars is de vrijheid en het niet zorgen hoeven te maken om type casting een enorm pluspunt. Deze manier van werken kan spijtig genoeg snel tot onverwachte bugs leiden, vaak onder de vorm van verstopte `undefined` errors. Dergelijke bugs ontstaan door dynamische typering en het moeten nakijken van types in de runtime. \\
TypeScript maakt JavaScript statisch getypeerd en zal types nakijken voor runtime, tijdens compiler time. Dit staat toe om `undefined` en mogelijke type errors preventief te achterhalen. \\
Bug preventie en statische typering zijn maar enkele voordelen die TypeScript biedt als toevoeging aan JavaScript.

\subsection{Code linting}

Waar TypeScript dient als taal veiligheid voor JavaScript is het nog steeds aan de ontwikkelaar zelf om zich te behoeden van mogelijke fouten in code. De meeste IDE's (Integrated Development Environment) houden wel rekening met typering in TypeScript en melden dat ook aan de ontwikkelaar. In deze tijd bevatten de meeste editors, zoals Visual Studio Code of WebStorm, een krachtige en toepasselijke IDE. \\ 
Voor het behouden van consistentie in de manier waarop code wordt geschreven binnen de service wordt er een `linter` geconfigureerd. Het geeft garantie tot code kwaliteit en het hanteren van specifieke best practices. De linter zorgt ervoor dat bepaalde `lint` regels gevolgd worden en past code formattering toe wanneer gespecificeerd.

\subsection{Node.js framework}

Net zoals bij de opkomst van JavaScript in de jaren 2000, is er bij de opkomst van Node.js ook onvermijdelijk markt ontstaan voor verscheidene frameworks gebaseerd op Node.js. In deze groei zijn er die dat zich focussen op het aanbieden van een volledig nieuwe omgeving op basis van bepaalde templates. Anderzijds zijn er ook frameworks die meer worden aangeboden als library, met als doeleinde een laag te vormen op de basis die Node.js vormt. Enkele voorbeelden hiervan zijn Express.js, Koa.js, Hapi.js, Restify,... \\
De frameworks die meer een toevoeging zijn in plaats van een inkapseling, brengen route afhandeling zonder de binding van een alles in één framework. Ze steunen op het principe van een `middleware stack`, wat kort door de bocht een verzameling is van functies die in een bepaalde volgorde worden uitgevoerd en daarbij logica of route manipulaties uitvoeren. Het manipuleren van de route houdt in dat er aanpassingen gedaan worden aan het binnenkomende verzoek of het uitgaande antwoord op de server.

\subsection{Database object mapping}

Bij het opbouwen van een RESTful service, met oog op data manipulatie en creatie (CRUD), wordt de overweging gemaakt welke soort databank zal gebruikt worden voor de `use case` in kwestie. Dit onderzoek focust zich niet op het soort databank en kiest simpelweg voor een relationele databank, namelijk MySQL. Waar het onderzoek zich wel op focust is hoe de verbinding wordt gelegd tussen de server en databank. \\
Ruwe data in databanken is niet geschikt om meteen te worden gebruikt door de Node.js server. De typering van data in de databank komt niet overeen met de JavaScript datatypes. Bij `data object mapping` wordt de ruwe data omgezet naar een bruikbaar JavaScript object met toepasselijke datatypes aan de hand van vooropgestelde data modellen. \\
Het onderzoek gebruikt een relationele databank en zal gebruik maken van een ORM (Object Relational Mapping) library.

\subsection{Documentatie}

Het documenteren van een API dient als wegwijzer voor gebruikers, andere teams binnen een organisatie of `third party` instanties die de API zouden gebruiken. De verschillende routes, of `endpoints`, worden beschreven met respectievelijk mogelijke data die mee over het verzoek wordt verstuurd of ontvangen via een antwoord. Zoals geldt voor een REST API worden alle mogelijke CRUD (Create Read Update Delete) operaties gedocumenteerd. \\
In organisaties met verschillende teams, die op zich verscheidene services beheren, wordt er nauw gelet op documentatie. Elk team is verantwoordelijk voor het aanpassen van de betreffende documentatie wanneer er, om het in vakjargon te zeggen, een `contract change` is. Dit kan duiden op naam verandering in van een endpoint of het toevoegen van een nieuwe.

%---------- Methodologie ------------------------------------------------------
\section{Methodologie}
\label{sec:methodologie}

Het onderzoek bestaat uit verschillende fasen, waarbij in de eerste fase een onderzoek gedaan wordt naar de verschillende mogelijke technologieën per service onderdeel. Dit kan bijvoorbeeld zijn voor bepalen van het Node.js of documentatie framework. De eerste fase maakt een modulair beeld van de volledige project omvang en zijn `tech stack`. \\
In een volgende fase wordt de basis gelegd van het project in Node.js en hoe TypeScript hier geïntegreerd kan worden. Daarbij wordt aangehaald welke configuratie er specifiek moet gebeuren en wat het juist inhoud. Verder in deze fase wordt voort gebouwd op de gelegde basis. Voor elk deel van de tech stack gaat een vergelijkende studie tussen de mogelijke technologieën uitmaken welke wordt toegevoegd. \\
Vervolgens volgt een test fase waarbij het geheel wordt gedraaid en getest. Het testen van de API zal gebeuren via Postman en mogelijks worden eigen gecreëerde applicaties gebruikt voor het genereren van API verkeer. Wanneer alles in het geheel getest is wordt over gegaan naar de laatste fase om conclusie te trekken over het gehele proces en de, al dan niet, beweegreden voor sommige keuzen.

%---------- Verwachte resultaten ----------------------------------------------
\section{Verwachte resultaten}
\label{sec:verwachte_resultaten}

Het opzetten van dergelijke server applicatie in Node.js met TypeScript integratie is gunstig voor het ontwikkelingsproces. De leercurve die er mee gepaard gaat is een kost die de ontwikkelaar of organisatie wilt dragen om meer zekerheid en taal veiligheid te hebben binnen Node.js services. \\
Na het onderzoeken van de geschikte `tech stack` voor REST API's in Node.js loont het om te zien dat de finale werking stabiel is bij hoog server verkeer. Het gecreëerde project kan dienen als basis voor uitwerkingen van backend services in bedrijven die een microservice architectuur hanteren.

%---------- Verwachte conclusies ----------------------------------------------
\section{Verwachte conclusies}
\label{sec:verwachte_conclusies}

Het opzetten van een Node.js server met Typescript zal niet altijd de go-to oplossing voor alle projecten. Alles hangt af van wat het project precies inhoud. Als er een kleine server moet opgezet worden die enkel moet dienen als proxy, zal niet de beslissing genomen worden om de volledige opzet te doen zoals beschreven in het onderzoek. Sommige projecten vragen een meer directe aanpak en daarvoor is puur JavaScript in Node.js meer rechtdoorzee. \\
Wanneer de aanpak zoals voorgesteld in het onderzoek kan worden gehanteerd is het vast en zeker een meerwaarde. Het gecreëerde project voldoet aan alle criteria die een moderne REST API in Node.js moet bieden met de meest prominente technologieën van het moment.



%------------------------------------------------------------------------------
% Referentielijst
%------------------------------------------------------------------------------
% TODO: de gerefereerde werken moeten in BibTeX-bestand ``voorstel.bib''
% voorkomen. Gebruik JabRef om je bibliografie bij te houden en vergeet niet
% om compatibiliteit met Biber/BibLaTeX aan te zetten (File > Switch to
% BibLaTeX mode)

\phantomsection
\printbibliography[heading=bibintoc]

\end{document}
