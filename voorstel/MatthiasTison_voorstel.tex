%==============================================================================
% Sjabloon onderzoeksvoorstel bachelorproef
%==============================================================================
% Gebaseerd op LaTeX-sjabloon ‘Stylish Article’ (zie voorstel.cls)
% Auteur: Jens Buysse, Bert Van Vreckem
%
% Compileren in TeXstudio:
%
% - Zorg dat Biber de bibliografie compileert (en niet Biblatex)
%   Options > Configure > Build > Default Bibliography Tool: "txs:///biber"
% - F5 om te compileren en het resultaat te bekijken.
% - Als de bibliografie niet zichtbaar is, probeer dan F5 - F8 - F5
%   Met F8 compileer je de bibliografie apart.
%
% Als je JabRef gebruikt voor het bijhouden van de bibliografie, zorg dan
% dat je in ``biblatex''-modus opslaat: File > Switch to BibLaTeX mode.

\documentclass{voorstel}

\usepackage{lipsum}

%------------------------------------------------------------------------------
% Metadata over het voorstel
%------------------------------------------------------------------------------

%---------- Titel & auteur ----------------------------------------------------

% TODO: geef werktitel van je eigen voorstel op
\PaperTitle{Titel voorstel}
\PaperType{Onderzoeksvoorstel Bachelorproef 2020-2021}

% TODO: vul je eigen naam in als auteur, geef ook je emailadres mee!
\Authors{Matthias Tison\textsuperscript{1}} % Authors
\CoPromotor{Arvid De Meyer\textsuperscript{2} (Codifly)}
\affiliation{\textbf{Contact:}
  \textsuperscript{1} \href{mailto:matthias.tison.w0715@student.hogent.be}{matthias.tison.w0715@student.hogent.be};
  \textsuperscript{2} \href{mailto:arvid.de.meyer@codifly.be}{arvid.de.meyer@codifly.be};
}

%---------- Abstract ----------------------------------------------------------

\Abstract{\\
    Dit werk draagt bij tot het opzetten van een basis voor stabiele JavaScript backend projecten in Node.js met TypeScript integratie. De opkomst van microservice architectuur in vele bedrijven reikt de hand naar het alom populaire Node.js als programmeer taal. Microservice principe steunt op het feit van meerdere onafhankelijke services in een cluster te hebben die samen een de gehele backend vormen. Dit principe roept om een consistente project structuur onder de services, zeker wanneer een microservice architectuur al gouw tien tot vijftien services kan omvatten. \\
    Het onderzoek bestaat eruit een stabiele en ontwikkelaars vriendelijke basis te leggen voor backend services in Node.js. Dit onderzoek focust zich op REST API's als backend service. Hierbij wordt de volledige `tech stack` bepaald en uitgelegd, met daarbij de nodige integratie in de project structuur. Verscheidene onderdelen van de `tech stack` worden bepaald op basis van vergelijking tussen verschillende technologieën. Volgend onderzoeksvoorstel stelt de voorstudie op van een backend oplossing in Node.js met Typescript integratie en wat verwacht wordt van de finale uitwerking. \\
    Uit dit onderzoek wordt getracht een robuuste, stabiele en ontwikkelaars vriendelijke omgeving op te zetten voor backend oplossingen in Node.js met Typescript integratie. Dit volgens de juiste principes voor het opzetten van dergelijke API, met daarbij de meest toepasselijke technologieën op basis van gegrond onderzoek. Aansluitend zou er CI  (continuous integration) en CD (continuous development) voorzien kunnen worden, zodat het release proces voor dergelijke projecten ook toevoeging heeft tot stabiliteit.
\\}

%---------- Onderzoeksdomein en sleutelwoorden --------------------------------
% TODO: Sleutelwoorden:
%
% Het eerste sleutelwoord beschrijft het onderzoeksdomein. Je kan kiezen uit
% deze lijst:
%
% - Mobiele applicatieontwikkeling
% - Webapplicatieontwikkeling
% - Applicatieontwikkeling (andere)
% - Systeembeheer
% - Netwerkbeheer
% - Mainframe
% - E-business
% - Databanken en big data
% - Machineleertechnieken en kunstmatige intelligentie
% - Andere (specifieer)
%
% De andere sleutelwoorden zijn vrij te kiezen

\Keywords{Applicatieontwikkeling -- Backend -- REST API -- Node.js -- TypeScript} % Keywords
\newcommand{\keywordname}{Sleutelwoorden} % Defines the keywords heading name

%---------- Titel, inhoud -----------------------------------------------------

\begin{document}

\flushbottom % Makes all text pages the same height
\maketitle % Print the title and abstract box
\tableofcontents % Print the contents section
\thispagestyle{empty} % Removes page numbering from the first page

%------------------------------------------------------------------------------
% Hoofdtekst
%------------------------------------------------------------------------------

% De hoofdtekst van het voorstel zit in een apart bestand, zodat het makkelijk
% kan opgenomen worden in de bijlagen van de bachelorproef zelf.
%---------- Inleiding ---------------------------------------------------------

\section{Introductie} % The \section*{} command stops section numbering
\label{sec:introductie}

Met haar snelle opkomst is de microservice-architectuur binnengedrongen in talrijke bedrijven. Vele slagen erin om deze strategie correct te implementeren, maar anderen lijden onder de nieuwe manier van werken met verschillende onafhankelijke services. \\
Het gebruik van de microservice-architectuur leunt er zich enerzijds goed toe het werk en het “ownership” van de verschillende services te verdelen onder teams. Elk team werkt op zijn eigen manier, onder idealiter een “agile” ontwikkelingsproces, aan de aan haar toegekende service. Een microservice-architectuur omvat anderzijds snel 10 of meer services, waardoor teams ook sneller een relatief groot aantal services moeten beheren. \\
Hieruit kan geconcludeerd worden dat er nood is aan het opbouwen van standaarden overheen teams. Het opleggen van een bepaalde standaard brengt een vorm van consistentie en meer consistentie leidt uiteindelijk tot stabiliteit. Dit onderzoek zal bijdragen tot het creëren van een doorgronde opzet als basis tot stabiele backend oplossingen in Node.js, met daarbij specifiek de focus op REST API als microservice. \\
Voor het doorvoeren van bepaalde standaarden moet er gegrond onderzoek gevoerd worden welke kan onderbouwen waarom de standaarden in kwestie gehanteerd worden. In dergelijk onderzoek kunnen volgende zaken afgevraagd worden: \\

\begin{itemize}
  \item Waarom wordt Node.js alom gebruikt voor serverapplicaties?
  \item In welke zin is een Typescript-integratie voordelig in Node.js-applicaties en hoe kan zo een integratie best gebeuren?
  \item Hoe kan “code linting” een bepaald niveau van codekwaliteit garanderen?
  \item Wat en welke zijn Node.js-libraries en kunnen toepasselijk zijn bij de uitwerking van een RESTful serverapplicatie?
  \item Waar wordt rekening mee gehouden bij het opzetten van een databankconnectie in Node.js-applicaties?
  \item In hoeverre wordt een API gedocumenteerd en wat zijn duidelijke en gebruiksvriendelijke documentatie-frameworks?
\end{itemize}

%---------- Stand van zaken ---------------------------------------------------

\section{Achtergrond}
\label{sec:achtergrond}

% Voor literatuurverwijzingen zijn er twee belangrijke commando's:
% \autocite{KEY} => (Auteur, jaartal) Gebruik dit als de naam van de auteur
%   geen onderdeel is van de zin.
% \textcite{KEY} => Auteur (jaartal)  Gebruik dit als de auteursnaam wel een
%   functie heeft in de zin (bv. ``Uit onderzoek door Doll & Hill (1954) bleek
%   ...'')

\subsection{Waarom Node.js?}

Node.js is een JavaScript runtime environment die het mogelijk maakt om de geliefde frontend taal ook naar de backend te brengen. \\
Sinds de eerste release in 2009 is Node.js steeds volwassener geworden en heeft het daarbij een standvaste community opgebouwd. Node.js is sinds enkele jaren één van de voornaamste standaarden voor backend-services, samen met klassieke populaire platformen zoals Java en Python.  \\
Met de opkomst van microservices biedt Node.js een geschikt platform door de makkelijke schaalbaarheid. Met native ondersteuning voor HTTP-requests leunt het zich perfect voor gebruik ter implementatie van een REST API. De native JSON serialisatie en deserialisatie biedt een veel gebruikte standaard voor een client/server-architectuur.
Dit onderzoek stelt een basis samen voor backend-applicaties en gebruikt daarvoor Node.js als backend-framework.

\subsection{TypeScript integratie}

JavaScript geeft de mogelijkheid om snel en vrij te ontwikkelen door zijn dynamisch typering, wat betekend dat de types van gedefinieerde variabelen niet gekend zijn tot in runtime. Naast dynamisch is JavaScript ook zwak getypeerd en betekend dat dat variabelen niet gebonden zijn tot één datatype. Dit staat toe om variabelen verschillende datatypes, zoals string of number, te geven doorheen de code. \\
Voor veel ontwikkelaars is de vrijheid en geen zorgen hoeven te maken om type-casting een enorm pluspunt. Deze manier van werken kan spijtig genoeg snel tot onverwachte bugs leiden, vaak onder de vorm van verstopte “undefined” errors. Dergelijke bugs ontstaan door dynamische typering en het moeten controleren van types in de runtime. \\
TypeScript maakt JavaScript statisch getypeerd en zal types nakijken vóór runtime, tijdens compiler time. Dit staat toe om “undefined” en mogelijke type-errors preventief op te vangen. \\
Bug preventie en statische typering zijn echter maar enkele voordelen die TypeScript biedt als toevoeging aan JavaScript. Voor het onderzoek wordt TypeScript geïntegreerd in een Node.js-applicatie en worden bepaalde principes en technische termen verder aangekaart.

\subsection{Code linting}

Waar TypeScript dient als taalveiligheid voor JavaScript is het nog steeds aan de ontwikkelaar om zich te behoeden van mogelijke fouten in de code. Meeste IDE's (Integrated Development Environment) houden wel rekening met typering in TypeScript en markeren type-errors in de editor. In deze tijd bevatten de meeste editors zoals Visual Studio Code of WebStorm een krachtige en toepasselijke IDE. \\ 
Voor het behouden van consistentie in de manier waarop code wordt geschreven binnen een project wordt er een “linter” geconfigureerd. Het geeft garantie tot codekwaliteit en het hanteren van specifieke ``best practices''. De linter zorgt ervoor dat bepaalde “lint”-regels gevolgd worden en past codeformattering toe wanneer gespecificeerd. \\
In dit onderzoek wordt een linter geconfigureerd door middel van de ES-lint library, met integratie van bijpassende lint-regels voor JavaScript en TypeScript.

\subsection{Frameworks en libraries in Node.js}

Net zoals bij de opkomst van JavaScript in de jaren 2000, is er bij de opkomst van Node.js ook onvermijdelijk een markt ontstaan voor verscheidene frameworks en libraries gebaseerd op Node.js. \\ 
In deze groei zijn er zij die zich focussen op het aanbieden van een volledig nieuw platform op basis van Node.js, met aansluitend bepaalde templates om een projectstructuur op te bouwen. Daarnaast zijn er zij die eerder worden aangeboden als library. De libraries hebben als doeleinde een laag te vormen op Node.js als platform, enkele voorbeelden hiervan zijn Express.js, Koa.js, Hapi.js, Restify,$\dots$ \\
De libraries, die meer een toevoeging zijn in plaats van een inkapseling, brengen route afhandeling zonder de binding van een alles in één framework. Ze steunen op het principe van een “middleware-stack”, wat kort door de bocht een verzameling is van functies die in een bepaalde volgorde worden uitgevoerd en daarbij logica of route manipulaties uitvoeren. Het manipuleren van de route houdt in dat er aanpassingen gedaan worden aan het binnenkomende request of het uitgaande response van de server. \\
Voor dit onderzoek wordt systematisch onderzocht hoe bepaalde Node.js libraries een aanwinst kunnen zijn voor de beste tech-stack die tracht aangetoond te worden.

\subsection{Database object mapping}

Bij het opbouwen van een RESTful service, met oog op manipulatie en creatie van data (CRUD), wordt de overweging gemaakt welke soort databank zal gebruikt worden voor de ``use case'' in kwestie. Dit onderzoek focust zich niet op het soort databank en kiest simpelweg voor een relationele databank, namelijk MySQL. Waar het onderzoek zich wel op focust is hoe de verbinding wordt opgesteld tussen de server en databank. \\
De ruwe data in databanken is niet geschikt om meteen te worden gebruikt door een Node.js server. De typering van data in de databank komt niet overeen met de JavaScript datatypes. Bij ``data object mapping'' wordt de ruwe data omgezet naar een bruikbaar JavaScript object met toepasselijke datatypes aan de hand van vooropgestelde datamodellen. \\
Het onderzoek gebruikt een relationele databank en zal gebruik maken van een ORM-library (Object Relational Mapping).

\subsection{Documentatie}

Het documenteren van een API dient als wegwijzer voor gebruikers, andere teams binnen een organisatie of ``third party'' instanties die de API zouden gebruiken. De verschillende routes, of ``endpoints'', worden beschreven met respectievelijk mogelijke data die mee over het request wordt verstuurd of ontvangen via een response. Elke HTTP-operatie die gebeurd op de server wordt gedocumenteerd. \\
In organisaties met verschillende teams, die op zich verscheidene services kunnen beheren, wordt er nauw gelet op documentatie. Elk team is verantwoordelijk voor het aanpassen van de betreffende documentatie wanneer er, om het in vakjargon te zeggen, een ``contract change'' is. Dit kan duiden op naamverandering in één of meer van de endpoints of het toevoegen van een nieuw endpoint. \\
Binnen het onderzoek wordt er aan de hand van een vergelijkende studie onderzocht welk documentatie-framework het best past op tech-stack.

%---------- Methodologie ------------------------------------------------------
\section{Methodologie}
\label{sec:methodologie}

Het onderzoek bestaat uit verschillende fasen, waarbij in de eerste fase een onderzoek gedaan wordt naar de verschillende mogelijke technologieën per serviceonderdeel. Dit kan bijvoorbeeld zijn voor het bepalen van een Node.js- of documentatie-framework. De eerste fase maakt een modulair beeld van het volledige project en zijn tech-stack. \\
In een volgende fase wordt de basis gelegd van het project in Node.js en hoe TypeScript geïntegreerd kan worden. Daarbij wordt aangehaald welke configuratie er specifiek moet gebeuren en wat het inhoud. Later in deze fase wordt verder gebouwd op de gelegde basis. Voor elk deel van de tech-stack gaat een vergelijkende studie tussen de mogelijke technologieën uitmaken welke wordt toegevoegd. \\
Vervolgens is er een testfase waarbij het gehele project wordt gedraaid en getest. Het testen van de API zal gebeuren via Postman en voor het genereren van API verkeer worden mogelijks eigen gecreëerde applicaties gebruikt. \\
Wanneer alles in het geheel getest is wordt overgegaan naar de laatste fase om een conclusie te trekken over het gehele proces en de, al dan niet, beweegredenen voor sommige keuzen.

%---------- Verwachte resultaten ----------------------------------------------
\section{Verwachte resultaten}
\label{sec:verwachte_resultaten}

Het opzetten van dergelijke serverapplicatie in Node.js met TypeScript integratie is gunstig voor het ontwikkelingsproces. De leercurve die er mee gepaard gaat is een kost die de ontwikkelaar of organisatie wilt dragen om meer zekerheid en taal veiligheid te hebben binnen Node.js-applicaties. \\
Na het onderzoeken van de geschikte tech-stack voor RESTful Node.js-services loont het om te zien dat de finale werking stabiel is bij hoog server verkeer. De gecreëerde basis kan dienen voor uitwerking van backend services in bedrijven die een microservice-architectuur hanteren.

%---------- Verwachte conclusies ----------------------------------------------
\section{Verwachte conclusies}
\label{sec:verwachte_conclusies}

De uitwerking die het onderzoek voorlegt voor Node.js backend oplossingen is geschikt als basis voor RESTful serverapplicaties. De TypeScript integratie draagt bij aan de stabiliteit van de service en voorziet taalveiligheid in een omgeving onderhevig aan ``silent error'' gevallen. Bovendien bestaat de tech-stack uit de meest prominente technologieën van het moment. \\
De opzet van een Node.js server met Typescript en de daarbij horende tech-stack is niet altijd de go-to oplossing voor alle projecten. Alles hangt af van projectinhoud. Als een kleine server moet opgezet worden die enkel als proxy moet dienen, zal bijvoorbeeld niet geopteerd worden om de volledige opzet te doen zoals beschreven in het onderzoek. Sommige projecten vragen een meer directe aanpak en daarvoor is een Node.je serverapplicatie in JavaScript meer rechtdoorzee.



%------------------------------------------------------------------------------
% Referentielijst
%------------------------------------------------------------------------------
% TODO: de gerefereerde werken moeten in BibTeX-bestand ``voorstel.bib''
% voorkomen. Gebruik JabRef om je bibliografie bij te houden en vergeet niet
% om compatibiliteit met Biber/BibLaTeX aan te zetten (File > Switch to
% BibLaTeX mode)

\phantomsection
\printbibliography[heading=bibintoc]

\end{document}
