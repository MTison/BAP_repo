%===============================================================================
% LaTeX sjabloon voor de bachelorproef toegepaste informatica aan HOGENT
% Meer info op https://github.com/HoGentTIN/bachproef-latex-sjabloon
%===============================================================================

\documentclass{bachproef-tin}

\usepackage{listings}
\usepackage{glossaries}
\usepackage{xcolor}
\usepackage{minted}
\usemintedstyle{vs}

\usepackage{hogent-thesis-titlepage} % Titelpagina conform aan HOGENT huisstijl

%%---------- Documenteigenschappen ---------------------------------------------

% De titel van het rapport/bachelorproef
\title{De systematische opbouw van een Node.js-applicatie met TypeScript-integratie als basis voor webservices}

% Je eigen naam
\author{Matthias Tison}

% De naam van je promotor (lector van de opleiding)
\promotor{Leen Vuyge}

% De naam van je co-promotor. Als je promotor ook je opdrachtgever is en je
% dus ook inhoudelijk begeleidt (en enkel dan!), mag je dit leeg laten.
\copromotor{Arvid De Meyer}

% Indien je bachelorproef in opdracht van/in samenwerking met een bedrijf of
% externe organisatie geschreven is, geef je hier de naam. Zoniet laat je dit
% zoals het is.
\instelling{Codifly}

% Academiejaar
\academiejaar{2020-2021}

% Examenperiode
%  - 1e semester = 1e examenperiode => 1
%  - 2e semester = 2e examenperiode => 2
%  - tweede zit  = 3e examenperiode => 3
\examenperiode{1}

% Genereer de `glossaries`
\makeglossaries

% Definities van de termen in de `glossary`
\newglossaryentry{API}{name=API, description={Application Programming Interface}}
\newglossaryentry{CSS}{name=CSS, description={Cascading Style Sheet}}
\newglossaryentry{ECMA}{name=ECMA, description={European Computer Manufacturers Association}}
\newglossaryentry{FIFO}{name=FIFO, description={First In First Out}}
\newglossaryentry{HTML}{name=HTML, description={HyperText Markup Language}}
\newglossaryentry{HTTP}{name=HTTP, description={Hypertext Transfer Protocol}}
\newglossaryentry{JIT}{name=JIT, description={Just In Time}}
\newglossaryentry{JSON}{name=JSON, description={JavaScript Object Notation}}
\newglossaryentry{LIFO}{name=LIFO, description={Last In First Out}}
\newglossaryentry{NPM}{name=npm, description={Node Package Manager}}
\newglossaryentry{OS}{name=OS, description={Operating System}}
\newglossaryentry{REST}{name=REST, description={Representational State Transfer}}
\newglossaryentry{XML}{name=XML, description={Extensible Markup Language}}

%===============================================================================
% Inhoud document
%===============================================================================

\begin{document}

%---------- Taalselectie -------------------------------------------------------
% Als je je bachelorproef in het Engels schrijft, haal dan onderstaande regel
% uit commentaar. Let op: de tekst op de voorkaft blijft in het Nederlands, en
% dat is ook de bedoeling!

%\selectlanguage{english}

%---------- Titelblad ----------------------------------------------------------
\inserttitlepage

%---------- Samenvatting, voorwoord --------------------------------------------
\usechapterimagefalse
%%=============================================================================
%% Voorwoord
%%=============================================================================

\chapter*{\IfLanguageName{dutch}{Woord vooraf}{Preface}}
\label{ch:voorwoord}

%% TODO:
%% Het voorwoord is het enige deel van de bachelorproef waar je vanuit je
%% eigen standpunt (``ik-vorm'') mag schrijven. Je kan hier bv. motiveren
%% waarom jij het onderwerp wil bespreken.
%% Vergeet ook niet te bedanken wie je geholpen/gesteund/... heeft


%%=============================================================================
%% Samenvatting
%%=============================================================================

% TODO: De "abstract" of samenvatting is een kernachtige (~ 1 blz. voor een
% thesis) synthese van het document.
%
% Deze aspecten moeten zeker aan bod komen:
% - Context: waarom is dit werk belangrijk?
% - Nood: waarom moest dit onderzocht worden?
% - Taak: wat heb je precies gedaan?
% - Object: wat staat in dit document geschreven?
% - Resultaat: wat was het resultaat?
% - Conclusie: wat is/zijn de belangrijkste conclusie(s)?
% - Perspectief: blijven er nog vragen open die in de toekomst nog kunnen
%    onderzocht worden? Wat is een mogelijk vervolg voor jouw onderzoek?
%
% LET OP! Een samenvatting is GEEN voorwoord!

%%---------- Nederlandse samenvatting -----------------------------------------
%
% TODO: Als je je bachelorproef in het Engels schrijft, moet je eerst een
% Nederlandse samenvatting invoegen. Haal daarvoor onderstaande code uit
% commentaar.
% Wie zijn bachelorproef in het Nederlands schrijft, kan dit negeren, de inhoud
% wordt niet in het document ingevoegd.

\IfLanguageName{english}{%
\selectlanguage{dutch}
\chapter*{Samenvatting}
\lipsum[1-4]
\selectlanguage{english}
}{}

%%---------- Samenvatting -----------------------------------------------------
% De samenvatting in de hoofdtaal van het document

\chapter*{\IfLanguageName{dutch}{Samenvatting}{Abstract}}

\lipsum[1-4]


%---------- Inhoudstafel -------------------------------------------------------
\pagestyle{empty} % Geen hoofding
\tableofcontents  % Voeg de inhoudstafel toe
\cleardoublepage  % Zorg dat volgende hoofstuk op een oneven pagina begint
\pagestyle{fancy} % Zet hoofding opnieuw aan

%---------- Lijst figuren, afkortingen, ... ------------------------------------

% Indien gewenst kan je hier een lijst van figuren/tabellen opgeven. Geef in
% dat geval je figuren/tabellen altijd een korte beschrijving:
%
%  \caption[korte beschrijving]{uitgebreide beschrijving}
%
% De korte beschrijving wordt gebruikt voor deze lijst, de uitgebreide staat bij
% de figuur of tabel zelf.

\listoffigures
\listoftables
% Print alle termen in de `glossary`
\printglossaries

%---------- Kern ---------------------------------------------------------------

% De eerste hoofdstukken van een bachelorproef zijn meestal een inleiding op
% het onderwerp, literatuurstudie en verantwoording methodologie.
% Aarzel niet om een meer beschrijvende titel aan deze hoofstukken te geven of
% om bijvoorbeeld de inleiding en/of stand van zaken over meerdere hoofdstukken
% te verspreiden!

%%=============================================================================
%% Inleiding
%%=============================================================================

\chapter{\IfLanguageName{dutch}{Inleiding}{Introduction}}
\label{ch:inleiding}

De inleiding moet de lezer net genoeg informatie verschaffen om het onderwerp te begrijpen en in te zien waarom de onderzoeksvraag de moeite waard is om te onderzoeken. In de inleiding ga je literatuurverwijzingen beperken, zodat de tekst vlot leesbaar blijft. Je kan de inleiding verder onderverdelen in secties als dit de tekst verduidelijkt. Zaken die aan bod kunnen komen in de inleiding~\autocite{Pollefliet2011}:

\begin{itemize}
  \item context, achtergrond
  \item afbakenen van het onderwerp
  \item verantwoording van het onderwerp, methodologie
  \item probleemstelling
  \item onderzoeksdoelstelling
  \item onderzoeksvraag
  \item \ldots
\end{itemize}

\section{\IfLanguageName{dutch}{Probleemstelling}{Problem Statement}}
\label{sec:probleemstelling}

Uit je probleemstelling moet duidelijk zijn dat je onderzoek een meerwaarde heeft voor een concrete doelgroep. De doelgroep moet goed gedefinieerd en afgelijnd zijn. Doelgroepen als ``bedrijven,'' ``KMO's,'' systeembeheerders, enz.~zijn nog te vaag. Als je een lijstje kan maken van de personen/organisaties die een meerwaarde zullen vinden in deze bachelorproef (dit is eigenlijk je steekproefkader), dan is dat een indicatie dat de doelgroep goed gedefinieerd is. Dit kan een enkel bedrijf zijn of zelfs één persoon (je co-promotor/opdrachtgever).

\section{\IfLanguageName{dutch}{Onderzoeksvraag}{Research question}}
\label{sec:onderzoeksvraag}

Wees zo concreet mogelijk bij het formuleren van je onderzoeksvraag. Een onderzoeksvraag is trouwens iets waar nog niemand op dit moment een antwoord heeft (voor zover je kan nagaan). Het opzoeken van bestaande informatie (bv. ``welke tools bestaan er voor deze toepassing?'') is dus geen onderzoeksvraag. Je kan de onderzoeksvraag verder specifiëren in deelvragen. Bv.~als je onderzoek gaat over performantiemetingen, dan 

\section{\IfLanguageName{dutch}{Onderzoeksdoelstelling}{Research objective}}
\label{sec:onderzoeksdoelstelling}

Wat is het beoogde resultaat van je bachelorproef? Wat zijn de criteria voor succes? Beschrijf die zo concreet mogelijk. Gaat het bv. om een proof-of-concept, een prototype, een verslag met aanbevelingen, een vergelijkende studie, enz.

\section{\IfLanguageName{dutch}{Opzet van deze bachelorproef}{Structure of this bachelor thesis}}
\label{sec:opzet-bachelorproef}

% Het is gebruikelijk aan het einde van de inleiding een overzicht te
% geven van de opbouw van de rest van de tekst. Deze sectie bevat al een aanzet
% die je kan aanvullen/aanpassen in functie van je eigen tekst.

De rest van deze bachelorproef is als volgt opgebouwd:

In Hoofdstuk~\ref{ch:stand-van-zaken} wordt een overzicht gegeven van de stand van zaken binnen het onderzoeksdomein, op basis van een literatuurstudie.

In Hoofdstuk~\ref{ch:methodologie} wordt de methodologie toegelicht en worden de gebruikte onderzoekstechnieken besproken om een antwoord te kunnen formuleren op de onderzoeksvragen.

% TODO: Vul hier aan voor je eigen hoofstukken, één of twee zinnen per hoofdstuk

In Hoofdstuk~\ref{ch:conclusie}, tenslotte, wordt de conclusie gegeven en een antwoord geformuleerd op de onderzoeksvragen. Daarbij wordt ook een aanzet gegeven voor toekomstig onderzoek binnen dit domein.
\chapter{\IfLanguageName{dutch}{Stand van zaken}{State of the art}}
\label{ch:stand-van-zaken}

% Tip: Begin elk hoofdstuk met een paragraaf inleiding die beschrijft hoe
% dit hoofdstuk past binnen het geheel van de bachelorproef. Geef in het
% bijzonder aan wat de link is met het vorige en volgende hoofdstuk.

% Pas na deze inleidende paragraaf komt de eerste sectiehoofding.

Dit hoofdstuk bevat je literatuurstudie. De inhoud gaat verder op de inleiding, maar zal het onderwerp van de bachelorproef *diepgaand* uitspitten. De bedoeling is dat de lezer na lezing van dit hoofdstuk helemaal op de hoogte is van de huidige stand van zaken (state-of-the-art) in het onderzoeksdomein. Iemand die niet vertrouwd is met het onderwerp, weet nu voldoende om de rest van het verhaal te kunnen volgen, zonder dat die er nog andere informatie moet over opzoeken \autocite{Pollefliet2011}.

Je verwijst bij elke bewering die je doet, vakterm die je introduceert, enz. naar je bronnen. In \LaTeX{} kan dat met het commando \texttt{$\backslash${textcite\{\}}} of \texttt{$\backslash${autocite\{\}}}. Als argument van het commando geef je de ``sleutel'' van een ``record'' in een bibliografische databank in het Bib\LaTeX{}-formaat (een tekstbestand). Als je expliciet naar de auteur verwijst in de zin, gebruik je \texttt{$\backslash${}textcite\{\}}.
Soms wil je de auteur niet expliciet vernoemen, dan gebruik je \texttt{$\backslash${}autocite\{\}}. In de volgende paragraaf een voorbeeld van elk.

\textcite{Knuth1998} schreef een van de standaardwerken over sorteer- en zoekalgoritmen. Experten zijn het erover eens dat cloud computing een interessante opportuniteit vormen, zowel voor gebruikers als voor dienstverleners op vlak van informatietechnologie~\autocite{Creeger2009}.

\lipsum[7-20]

%%=============================================================================
%% Methodologie
%%=============================================================================

\chapter{\IfLanguageName{dutch}{Methodologie}{Methodology}}
\label{ch:methodologie}

%% TODO: Hoe ben je te werk gegaan? Verdeel je onderzoek in grote fasen, en
%% licht in elke fase toe welke stappen je gevolgd hebt. Verantwoord waarom je
%% op deze manier te werk gegaan bent. Je moet kunnen aantonen dat je de best
%% mogelijke manier toegepast hebt om een antwoord te vinden op de
%% onderzoeksvraag.

\lipsum[21-25]



%\input{...}

%%=============================================================================
%% Conclusie
%%=============================================================================

\chapter{Conclusie}
\label{ch:conclusie}

% TODO: Trek een duidelijke conclusie, in de vorm van een antwoord op de
% onderzoeksvra(a)g(en). Wat was jouw bijdrage aan het onderzoeksdomein en
% hoe biedt dit meerwaarde aan het vakgebied/doelgroep? 
% Reflecteer kritisch over het resultaat. In Engelse teksten wordt deze sectie
% ``Discussion'' genoemd. Had je deze uitkomst verwacht? Zijn er zaken die nog
% niet duidelijk zijn?
% Heeft het onderzoek geleid tot nieuwe vragen die uitnodigen tot verder 
%onderzoek?

\lipsum[76-80]



%%=============================================================================
%% Bijlagen
%%=============================================================================

\appendix
\renewcommand{\chaptername}{Appendix}

%%---------- Onderzoeksvoorstel -----------------------------------------------

\chapter{Onderzoeksvoorstel}

Het onderwerp van deze bachelorproef is gebaseerd op een onderzoeksvoorstel dat vooraf werd beoordeeld door de promotor. Dat voorstel is opgenomen in deze bijlage.

% Verwijzing naar het bestand met de inhoud van het onderzoeksvoorstel
%---------- Inleiding ---------------------------------------------------------

\section{Introductie} % The \section*{} command stops section numbering
\label{sec:introductie}

%\begin{itemize}
%  \item de probleemstelling en context
%  \item de motivatie en relevantie voor het onderzoek
%  \item de doelstelling en onderzoeksvraag/-vragen
%\end{itemize}

Het opkomen van de microservice architectuur heeft dit ook doen binnen dringen in talrijke bedrijven. Vele slagen erin om deze strategie correct te implementeren, maar vele lijden ook onder de nieuwe manier van werken met verschillende onafhankelijke services.
Het gebruik van dit soort architectuur leunt er zich ook toe om het werk en `ownership` van de verschillende services te verdelen onder teams. Elk team werkt op zijn eigen manier onder, idealiter, een `agile` ontwikkelingsproces aan de toegekende service. Een microservice architectuur omvat ook snel 10 tot 15 services, waardoor teams ook sneller een bepaald aantal van de services beheren. \\
Hieruit kan geconcludeerd worden dat er nood is aan het opbouwen van standaarden over heen teams. Het opleggen van een bepaalde standaard brengt een vorm van consistentie en meer consistentie lijdt uiteindelijk tot stabiliteit. Dit onderzoek zal bijdragen tot het creëren van een doorgronde opzet als basis tot stabiele backend services in Node.js, met daarbij specifiek de focus op REST API als microservice. \\
Voor het doorvoeren van bepaalde standaarden moet er gegrond onderzoek gevoerd worden die kan onderbouwen waarom de standaard in kwestie gehanteerd wordt. Hierbij kan afgevraagd worden waarom een Typescript integratie benefiet zou hebben in Node.js en hoe zo een integratie gebeurd? Welke stappen worden uitgevoerd om een bepaald niveau van `code quality` te bekomen? Wat zijn de meest prominente technologieën als `tech stack` voor REST API in Node.js? Hoe kunnen REST principes correct toegepast worden in correlatie met de `tech stack` en projectstructuur?

%---------- Stand van zaken ---------------------------------------------------

\section{State-of-the-art}
\label{sec:state-of-the-art}

% Voor literatuurverwijzingen zijn er twee belangrijke commando's:
% \autocite{KEY} => (Auteur, jaartal) Gebruik dit als de naam van de auteur
%   geen onderdeel is van de zin.
% \textcite{KEY} => Auteur (jaartal)  Gebruik dit als de auteursnaam wel een
%   functie heeft in de zin (bv. ``Uit onderzoek door Doll & Hill (1954) bleek
%   ...'')

\subsection{Waarom Node.js?}

Node.js is sinds enkele jaren één van de standaarden voor backend services, samen met grootheden zoals Java en Python. Sinds de eerst release in 2009 is het steeds volwassener geworden en heeft daarbij een stand vaste community opgebouwd. De JavaScript runtime environment maakt het mogelijk om de geliefde frontend web taal ook naar de backend te brengen. \\
Met de opkomst van microservices biedt Node.js een geschikte omgeving. Door de makkelijke schaalbaarheid en native ondersteuning voor HTTP aanvragen leunt het zich perfect voor gebruik als REST API. Met native JSON serialisatie en de-serialisatie biedt het een veel gebruikte standaard voor client/server-architectuur.

\subsection{TypeScript integratie}

JavaScript geeft de mogelijkheid om snel en vrij te ontwikkelen door zijn dynamisch typering, wat betekend dat de types van gedefinieerde variabelen niet gekend zijn tot in runtime. Naast dynamisch is JavaScript ook zwak getypeerd en betekend dat variabelen niet gebonden zijn tot één data type. Dit staat toe om variabelen verschillende data types, zoals string of number, te geven doorheen de code. \\
Voor veel ontwikkelaars is de vrijheid en het niet zorgen hoeven te maken om type casting een enorm pluspunt. Deze manier van werken kan spijtig genoeg snel tot onverwachte bugs leiden, vaak onder de vorm van verstopte `undefined` errors. Dergelijke bugs ontstaan door dynamische typering en het moeten nakijken van types in de runtime. \\
TypeScript maakt JavaScript statisch getypeerd en zal types nakijken voor runtime, tijdens compiler time. Dit staat toe om `undefined` en mogelijke type errors preventief te achterhalen. \\
Bug preventie en statische typering zijn maar enkele voordelen die TypeScript biedt als toevoeging aan JavaScript.

\subsection{Code linting}

Waar TypeScript dient als taal veiligheid voor JavaScript is het nog steeds aan de ontwikkelaar zelf om zich te behoeden van mogelijke fouten in code. De meeste IDE's (Integrated Development Environment) houden wel rekening met typering in TypeScript en melden dat ook aan de ontwikkelaar. In deze tijd bevatten de meeste editors, zoals Visual Studio Code of WebStorm, een krachtige en toepasselijke IDE. \\ 
Voor het behouden van consistentie in de manier waarop code wordt geschreven binnen de service wordt er een `linter` geconfigureerd. Het geeft garantie tot code kwaliteit en het hanteren van specifieke best practices. De linter zorgt ervoor dat bepaalde `lint` regels gevolgd worden en past code formattering toe wanneer gespecificeerd.

\subsection{Node.js framework}

Net zoals bij de opkomst van JavaScript in de jaren 2000, is er bij de opkomst van Node.js ook onvermijdelijk markt ontstaan voor verscheidene frameworks gebaseerd op Node.js. In deze groei zijn er die dat zich focussen op het aanbieden van een volledig nieuwe omgeving op basis van bepaalde templates. Anderzijds zijn er ook frameworks die meer worden aangeboden als library, met als doeleinde een laag te vormen op de basis die Node.js vormt. Enkele voorbeelden hiervan zijn Express.js, Koa.js, Hapi.js, Restify,... \\
De frameworks die meer een toevoeging zijn in plaats van een inkapseling, brengen route afhandeling zonder de binding van een alles in één framework. Ze steunen op het principe van een `middleware stack`, wat kort door de bocht een verzameling is van functies die in een bepaalde volgorde worden uitgevoerd en daarbij logica of route manipulaties uitvoeren. Het manipuleren van de route houdt in dat er aanpassingen gedaan worden aan het binnenkomende verzoek of het uitgaande antwoord op de server.

\subsection{Database object mapping}

Bij het opbouwen van een RESTful service, met oog op data manipulatie en creatie (CRUD), wordt de overweging gemaakt welke soort databank zal gebruikt worden voor de `use case` in kwestie. Dit onderzoek focust zich niet op het soort databank en kiest simpelweg voor een relationele databank, namelijk MySQL. Waar het onderzoek zich wel op focust is hoe de verbinding wordt gelegd tussen de server en databank. \\
Ruwe data in databanken is niet geschikt om meteen te worden gebruikt door de Node.js server. De typering van data in de databank komt niet overeen met de JavaScript datatypes. Bij `data object mapping` wordt de ruwe data omgezet naar een bruikbaar JavaScript object met toepasselijke datatypes aan de hand van vooropgestelde data modellen. \\
Het onderzoek gebruikt een relationele databank en zal gebruik maken van een ORM (Object Relational Mapping) library.

\subsection{Documentatie}

Het documenteren van een API dient als wegwijzer voor gebruikers, andere teams binnen een organisatie of `third party` instanties die de API zouden gebruiken. De verschillende routes, of `endpoints`, worden beschreven met respectievelijk mogelijke data die mee over het verzoek wordt verstuurd of ontvangen via een antwoord. Zoals geldt voor een REST API worden alle mogelijke CRUD (Create Read Update Delete) operaties gedocumenteerd. \\
In organisaties met verschillende teams, die op zich verscheidene services beheren, wordt er nauw gelet op documentatie. Elk team is verantwoordelijk voor het aanpassen van de betreffende documentatie wanneer er, om het in vakjargon te zeggen, een `contract change` is. Dit kan duiden op naam verandering in van een endpoint of het toevoegen van een nieuwe.

%---------- Methodologie ------------------------------------------------------
\section{Methodologie}
\label{sec:methodologie}

Het onderzoek bestaat uit verschillende fasen, waarbij in de eerste fase een onderzoek gedaan wordt naar de verschillende mogelijke technologieën per service onderdeel. Dit kan bijvoorbeeld zijn voor bepalen van het Node.js of documentatie framework. De eerste fase maakt een modulair beeld van de volledige project omvang en zijn `tech stack`. \\
In een volgende fase wordt de basis gelegd van het project in Node.js en hoe TypeScript hier geïntegreerd kan worden. Daarbij wordt aangehaald welke configuratie er specifiek moet gebeuren en wat het juist inhoud. Verder in deze fase wordt voort gebouwd op de gelegde basis. Voor elk deel van de tech stack gaat een vergelijkende studie tussen de mogelijke technologieën uitmaken welke wordt toegevoegd. \\
Vervolgens volgt een test fase waarbij het geheel wordt gedraaid en getest. Het testen van de API zal gebeuren via Postman en mogelijks worden eigen gecreëerde applicaties gebruikt voor het genereren van API verkeer. Wanneer alles in het geheel getest is wordt over gegaan naar de laatste fase om conclusie te trekken over het gehele proces en de, al dan niet, beweegreden voor sommige keuzen.

%---------- Verwachte resultaten ----------------------------------------------
\section{Verwachte resultaten}
\label{sec:verwachte_resultaten}

Het opzetten van dergelijke server applicatie in Node.js met TypeScript integratie is gunstig voor het ontwikkelingsproces. De leercurve die er mee gepaard gaat is een kost die de ontwikkelaar of organisatie wilt dragen om meer zekerheid en taal veiligheid te hebben binnen Node.js services. \\
Na het onderzoeken van de geschikte `tech stack` voor REST API's in Node.js loont het om te zien dat de finale werking stabiel is bij hoog server verkeer. Het gecreëerde project kan dienen als basis voor uitwerkingen van backend services in bedrijven die een microservice architectuur hanteren.

%---------- Verwachte conclusies ----------------------------------------------
\section{Verwachte conclusies}
\label{sec:verwachte_conclusies}

Het opzetten van een Node.js server met Typescript zal niet altijd de go-to oplossing voor alle projecten. Alles hangt af van wat het project precies inhoud. Als er een kleine server moet opgezet worden die enkel moet dienen als proxy, zal niet de beslissing genomen worden om de volledige opzet te doen zoals beschreven in het onderzoek. Sommige projecten vragen een meer directe aanpak en daarvoor is puur JavaScript in Node.js meer rechtdoorzee. \\
Wanneer de aanpak zoals voorgesteld in het onderzoek kan worden gehanteerd is het vast en zeker een meerwaarde. Het gecreëerde project voldoet aan alle criteria die een moderne REST API in Node.js moet bieden met de meest prominente technologieën van het moment.



%%---------- Andere bijlagen --------------------------------------------------
% TODO: Voeg hier eventuele andere bijlagen toe
%\input{...}

%%---------- Referentielijst --------------------------------------------------

\printbibliography[heading=bibintoc]

\end{document}
